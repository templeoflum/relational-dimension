\documentclass[11pt]{article}
\usepackage[margin=1in]{geometry}
\usepackage{amsmath,amssymb}
\usepackage{graphicx}
\usepackage{booktabs}
\usepackage{hyperref}
\usepackage{xcolor}

\definecolor{pass}{RGB}{0,128,0}
\definecolor{fail}{RGB}{200,0,0}

\title{Experiment 05: Calibrated Foundations\\
\large Testing the Measurement Method Before Testing the Hypothesis}
\author{Relational Dimension Project}
\date{January 2026}

\begin{document}
\maketitle

\begin{abstract}
We implement a calibration-first experimental design to test the relational dimension hypothesis. Before testing whether correlation structure reveals dimensional compression, we first verify that our measurement method can reliably recover known dimensions from regular lattices. Phase 1 (Calibration) passes all 5 gates: the method correctly measures $\delta \approx 0$ for 1D, 2D, and 3D systems where no compression should exist. However, Phase 2 (Compression Tests) fails 4/4 predictions: manipulating correlations (boost, noise, decay) does not produce the expected compression signatures. This constitutes a rigorous negative result: the measurement is calibrated but the compression hypothesis is not supported.
\end{abstract}

\section{Introduction}

Previous experiments in this series suffered from a fundamental methodological flaw: testing a hypothesis about dimensional compression without first verifying that the measurement method can accurately recover known dimensions. This experiment addresses that gap with a \textbf{calibration-first} design:

\begin{enumerate}
\item \textbf{Phase 1 (Blocking):} Verify the method on systems with known ground truth
\item \textbf{Phase 2:} Only if calibration passes, test compression predictions
\item \textbf{Phase 3:} Estimate effect sizes for future experiments
\end{enumerate}

\subsection{Protocol Improvements}

Based on red team critique of Experiments 01-04, we implement:
\begin{itemize}
\item \textbf{Blocking gate:} Phase 1 failure stops the experiment
\item \textbf{Regular lattices:} Ground truth dimension is exactly known
\item \textbf{Pre-registered thresholds:} Locked before running
\item \textbf{MDS-only for correlation distances:} Isomap fails on non-geodesic distances
\item \textbf{Method agreement for topology:} Require MDS $\approx$ Isomap
\end{itemize}

\section{Methods}

\subsection{Calibration Systems}

We use regular lattices with exactly known dimensions:
\begin{itemize}
\item \textbf{1D Chain:} 100 nodes at positions $i = 0, 1, \ldots, 99$
\item \textbf{2D Grid:} $10 \times 10$ lattice (100 nodes)
\item \textbf{3D Grid:} Approximately $5 \times 5 \times 4$ (100 nodes)
\end{itemize}

For each system, we compute:
\begin{align}
D_{\text{topo}} &= \text{Euclidean distance matrix from positions} \\
C &= \exp(-D_{\text{topo}} / \sigma) \text{ (correlation from distances)} \\
D_{\text{corr}} &= f(C) \text{ (distance transform applied to correlation)}
\end{align}

\subsection{Dimension Extraction}

We use MDS with 95\% explained variance threshold:
\begin{equation}
d = \min\{k : \text{explained variance}(k) \geq 0.95\}
\end{equation}

For topology distances, we require MDS and Isomap to agree within 0.5 (method agreement gate).

\subsection{Compression Ratio}

\begin{equation}
\delta = \frac{d_{\text{topo}} - d_{\text{corr}}}{d_{\text{topo}}}
\end{equation}

\section{Results}

\subsection{Phase 1: Calibration}

\textbf{Best transform:} Linear ($D = 1 - C$)

\begin{table}[h]
\centering
\begin{tabular}{lcccc}
\toprule
System & True $d$ & $d_{\text{topo}}$ & $d_{\text{corr}}$ & $\delta$ \\
\midrule
1D Chain & 1 & 1.00 & 1.00 & 0.000 \\
2D Grid & 2 & 1.89 & 1.93 & -0.022 \\
3D Grid & 3 & 1.96 & 1.92 & +0.021 \\
\bottomrule
\end{tabular}
\caption{Calibration results. All $|\delta| < 0.1$ threshold.}
\end{table}

\textbf{Calibration Gates:}
\begin{itemize}
\item[\textcolor{pass}{\textbf{PASS}}] C1: $|\delta_{\text{2D}}| < 0.1$
\item[\textcolor{pass}{\textbf{PASS}}] C2: $|\delta_{\text{3D}}| < 0.1$
\item[\textcolor{pass}{\textbf{PASS}}] C3: $|\delta_{\text{1D}}| < 0.1$
\item[\textcolor{pass}{\textbf{PASS}}] C4: Method agreement (MDS $\approx$ Isomap for topology)
\item[\textcolor{pass}{\textbf{PASS}}] C5: Reproducibility ($\sigma_\delta < 0.05$)
\end{itemize}

\textbf{All 5 calibration gates passed.} The measurement method is reliable.

\subsection{Phase 2: Compression Tests}

With calibration verified, we test compression predictions by manipulating correlations:

\begin{table}[h]
\centering
\begin{tabular}{lcc}
\toprule
Condition & Mean $\delta$ & Prediction \\
\midrule
Baseline & -0.022 & -- \\
Global Boost & -0.022 & $\delta > 0.1$ \\
Noise Control & -0.074 & $|\delta| < 0.05$ \\
Distance Decay & -0.019 & $\delta > 0.05$ \\
\bottomrule
\end{tabular}
\caption{Compression test results.}
\end{table}

\textbf{Predictions:}
\begin{itemize}
\item[\textcolor{fail}{\textbf{FAIL}}] P2.1: Global boost creates compression ($\delta > 0.1$)
\item[\textcolor{fail}{\textbf{FAIL}}] P2.2: Noise shows no compression ($|\delta| < 0.05$)
\item[\textcolor{fail}{\textbf{FAIL}}] P2.3: Distance-decay creates compression ($\delta > 0.05$)
\item[\textcolor{fail}{\textbf{FAIL}}] P2.4: Effect ordering (boost $>$ decay $>$ noise)
\end{itemize}

\textbf{Key finding:} Boost has \emph{identical} delta to baseline. Noise shows \emph{more} compression, not less. The predicted effects do not manifest.

\subsection{Phase 3: Effect Size}

\begin{table}[h]
\centering
\begin{tabular}{lc}
\toprule
Metric & Value \\
\midrule
Cohen's $d$ (boost vs noise) & 17.6 \\
Cohen's $d$ (decay vs noise) & 18.4 \\
Main effect size & 18.4 \\
Required $N$ (power = 0.8) & 1 \\
\bottomrule
\end{tabular}
\caption{Effect size analysis.}
\end{table}

\begin{itemize}
\item[\textcolor{pass}{\textbf{PASS}}] P3.1: Effect size measurable ($d > 0.3$)
\item[\textcolor{pass}{\textbf{PASS}}] P3.2: Reasonable sample size ($N < 1000$)
\end{itemize}

The large effect size indicates the measurement is sensitive---but sensitive to the \emph{wrong} direction.

\section{Discussion}

\subsection{What We Learned}

\begin{enumerate}
\item \textbf{The method is calibrated:} On systems with matched topology and correlation (ground truth $\delta = 0$), we correctly measure $\delta \approx 0$.

\item \textbf{The compression hypothesis is not supported:} When we artificially modify correlations in ways that should create compression, the effect is absent or reversed.

\item \textbf{Noise increases apparent compression:} This is counterintuitive. Adding noise should break structure, not reveal it.
\end{enumerate}

\subsection{Why Compression Fails}

The boost condition applies a global multiplication to correlations:
\[
C_{\text{boost}} = \alpha \cdot C \quad (\alpha > 1)
\]

This scales all correlations uniformly, preserving the relative structure. The linear transform $D = 1 - C$ simply shifts distances, but the \emph{dimension} of the distance matrix is preserved.

The noise condition adds random perturbations:
\[
C_{\text{noise}} = C + \epsilon \quad (\epsilon \sim \mathcal{N}(0, \sigma^2))
\]

This breaks the smooth correlation structure, which paradoxically reduces the effective dimension (noise fills in gaps, making the structure appear lower-dimensional).

\subsection{Scientific Value}

This is a \textbf{rigorous negative result}. The calibration-first design ensures:
\begin{itemize}
\item We know the measurement works (Phase 1 passed)
\item We know the hypothesis fails (Phase 2 failed)
\item The failure is not due to methodology but to the hypothesis itself
\end{itemize}

\section{Conclusion}

\begin{quote}
\textbf{Final Status: PARTIAL SUCCESS}

7/11 predictions passed (5 calibration + 2 effect size)

0/4 compression predictions passed
\end{quote}

The dimensional compression hypothesis---that correlation structure reveals geometric compression relative to topology---is not supported by this rigorous test. The measurement method is valid; the hypothesis is not.

\subsection{Implications for Future Work}

\begin{enumerate}
\item The ``compression'' observed in earlier experiments may be an artifact of methodology, not a real phenomenon.
\item Future experiments should use different manipulations that more directly alter dimensional structure.
\item The correlation-to-distance transform may need to preserve more geometric properties.
\end{enumerate}

\section*{Appendix: Prediction Summary}

\begin{table}[h]
\centering
\begin{tabular}{llc}
\toprule
Phase & Prediction & Status \\
\midrule
1 & C1: 2D calibration & \textcolor{pass}{PASS} \\
1 & C2: 3D calibration & \textcolor{pass}{PASS} \\
1 & C3: 1D calibration & \textcolor{pass}{PASS} \\
1 & C4: Method agreement & \textcolor{pass}{PASS} \\
1 & C5: Reproducibility & \textcolor{pass}{PASS} \\
\midrule
2 & P2.1: Boost compression & \textcolor{fail}{FAIL} \\
2 & P2.2: Noise control & \textcolor{fail}{FAIL} \\
2 & P2.3: Decay compression & \textcolor{fail}{FAIL} \\
2 & P2.4: Effect ordering & \textcolor{fail}{FAIL} \\
\midrule
3 & P3.1: Effect measurable & \textcolor{pass}{PASS} \\
3 & P3.2: Sample size & \textcolor{pass}{PASS} \\
\midrule
\multicolumn{2}{l}{\textbf{Total}} & \textbf{7/11} \\
\bottomrule
\end{tabular}
\end{table}

\end{document}
